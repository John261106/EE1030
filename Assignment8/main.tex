%iffalse
\let\negmedspace\undefined
\let\negthickspace\undefined
\documentclass[journal,12pt,onecolumn]{IEEEtran}
\usepackage{cite}
\usepackage{amsmath,amssymb,amsfonts,amsthm}
\usepackage{algorithmic}
\usepackage{multicol}
\usepackage{graphicx}
\usepackage{textcomp}
\usepackage{xcolor}
\usepackage{txfonts}
\usepackage{listings}
\usepackage{enumitem}
\usepackage{mathtools}
\usepackage{gensymb}
\usepackage{comment}
\usepackage[breaklinks=true]{hyperref}
\usepackage{tkz-euclide} 
\usepackage{listings}
\usepackage{gvv}                                        
%\def\inputGnumericTable{}                                 
\usepackage[latin1]{inputenc}                                
\usepackage{color}                                            
\usepackage{array}                                            
\usepackage{longtable}                                       
\usepackage{calc}                                             
\usepackage{multirow}                                         
\usepackage{hhline}                                           
\usepackage{ifthen}                                           
\usepackage{lscape}
\usepackage{tabularx}
\usepackage{array}
\usepackage{float}
\newtheorem{theorem}{Theorem}[section]
\newtheorem{problem}{Problem}
\newtheorem{proposition}{Proposition}[section]
\newtheorem{lemma}{Lemma}[section]
\newtheorem{corollary}[theorem]{Corollary}
\newtheorem{example}{Example}[section]
\newtheorem{definition}[problem]{Definition}
\newcommand{\BEQA}{\begin{eqnarray}}
\newcommand{\EEQA}{\end{eqnarray}}
\newcommand{\define}{\stackrel{\triangle}{=}}
\theoremstyle{remark}
\newtheorem{rem}{Remark}

% Marks the beginning of the document
\begin{document}
\bibliographystyle{IEEEtran}
\vspace{3cm}

\title{\textbf{GATE 2007 MA}}
\author{EE24BTECH11032- John Bobby}
\maketitle
\bigskip

\renewcommand{\thefigure}{\theenumi}
\renewcommand{\thetable}{\theenumi}
\setlength{\columnsep}{2.5em}
\begin{enumerate}
    \item Let $f\brak{z}=2z^2-1$. Then the maximum value of $\abs{f\brak{z}}$ on the unit disc D=$\cbrak{z \in \vec{C}:\abs{z}<1}$ equals
    \begin{enumerate}
        \item $1$
        \item $2$
        \item $3$
        \item $4$
    \end{enumerate}
    \item Let $f\brak{z}=\frac{1}{z^2-3z+2}$ Then the coefficient of $\frac{1}{z^3}$ in the Laurent series expansion of $f\brak{x}$ for $\abs{z}>1$ is 
    \begin{enumerate}
        \item $0$
        \item $1$
        \item $3$
        \item $5$
    \end{enumerate}
    \item Let f:$\vec{C} \to \vec{C}$ be an arbitrary analytic function satisfying f$\brak{0}=0$ and $f\brak{1}=2$. Then
    \begin{enumerate}
        \item there exists a sequence $\cbrak{z_n}$ such that $\abs{z_n}>n$ and $\abs{f\brak{z_n}}>n$
        \item there exists a sequence $\cbrak{z_n}$ such that $\abs{z_n}>n$ and $\abs{f\brak{z_n}}<n$
        \item there exists a sequence $\cbrak{z_n}$ such that $\abs{f\brak{z_n}}>n$
        \item there exists a sequence $\cbrak{z_n}$ such that $z_n \to 0$ and $f\brak{z_n} \to 2$
    \end{enumerate}
    \item Define f:$\vec{C} \to \vec{C}$ by \\
    $f\brak{z}=\left\{ \begin{array}{ll} 0,\quad if \quad Re\brak{z}=0 \quad or \quad Im\brak{z}=0,\\ z, \quad otherwise \end{array} \right. $\\
    Then the set of points where f is analytic is 
    \begin{enumerate}
        \item $\cbrak{z:Re\brak{z} \neq 0  \quad and \quad Im\brak{z} \neq 0}$
        \item $\cbrak{z:Re\brak{z} \neq 0}$
        \item $\cbrak{z:Re\brak{z} \neq 0  \quad or \quad Im\brak{z} \neq 0}$
        \item $\cbrak{z:Im\brak{z} \neq 0}$
    \end{enumerate}
    \item Let U$\brak{n}$ be the set of all positive integers less than $n$ and relatively prime to $n$. Then U$\brak{n}$ is the group under multiplication modulo $n$. For $n=248$, the number of elements in U$\brak{n}$ is 
    \begin{enumerate}
        \item $60$
        \item $120$
        \item $180$
        \item $240$
    \end{enumerate}
    \item Let R$\sbrak{x}$ be the polynomial ring in $x$ with real coefficients and let $I=\langle x^2+1\rangle$ be the ideal generated by the polynomial $x^2+1$ in R$\sbrak{x}$. Then
    \begin{enumerate}
        \item $I$ is a maximum ideal
        \item $I$ is a prime ideal but NOT a maximum ideal
        \item $I$ is NOT a prime ideal
        \item R$\sbrak{x}$/$I$ has zero divisors
    \end{enumerate}
    \item Consider $\vec{Z}_5$ and $\vec{Z}_{20}$ as rings modulo $5$ and $20$, respectively. Then the number of homomorphisms $\varphi : \vec{Z_5} \to \vec{Z_{20}}$ is 
    \begin{enumerate}
        \item $1$
        \item $2$
        \item $4$
        \item $5$
    \end{enumerate}
    \item Let $\vec{Q}$ be the field of rational numbers and consider $\vec{Z_2}$ as a field modulo 2. Let $f\brak{x}=x^3-9x^2+9x+3$. Then $f\brak{x}$ is
    \begin{enumerate}
        \item irreducible over $\vec{Q}$ but reducible over $\vec{Z_2}$
        \item irreducible over both $\vec{Q}$ and $\vec{Z_2}$
        \item reducible over $\vec{Q}$ but irreducible over $\vec{Z_2}$
        \item reducible over both $\vec{Q}$ and $\vec{Z_2}$
    \end{enumerate}
    \item Consider $\vec{Z_5}$ as a field modulo $5$ and let $f\brak{x}=x^5+4x^4+4x^3+4x^2+x+1$. Then the zeroes of $f\brak{x}$ over $\vec{Z_5}$ are $1$ and $3$ with respective multiplicity
    \begin{enumerate}
        \item $1$ and $4$
        \item $2$ and $3$
        \item $2$ and $2$
        \item $1$ and $2$
    \end{enumerate}
    \item Consider the Hilbert space $l^2=\cbrak{\vec{x}=\cbrak{x_n}:x_n \in \vec{R},\sum_{n=1}^{\infty}{x_n}^2<\infty}$. Let $E=\cbrak{\cbrak{x_n}:\abs{x_n}\leq\frac{1}{n}\quad\text{for all n}}$ be a subset of $l^2$. Then
    \begin{enumerate}
        \item $E^o=\cbrak{\vec{x}:\abs{x_n}<\frac{1}{n}\quad \text{for all n}}$
        \item $E^o=E$
        \item $E^o=\cbrak{\vec{x}:\abs{x_n}<\frac{1}{n}\quad \text{for all n but finitely many n}}$
        \item $E^o=\phi$
    \end{enumerate}
    \item Let $X$ and $Y$ be normed linear spaces and let $T:X \to Y$ be a linear map. Then $T$ is contiuous if 
    \begin{enumerate}
        \item $Y$ is finite dimensional
        \item $X$ is finite dimensional
        \item $T$ is one to one
        \item $T$ is onto
    \end{enumerate}
    \item Let $X$ be a normed linear space and let $E_1,E_2 \subseteq X$. Define $E_1+E_2=\cbrak{x+y:x \in E_1,y\in E_2}$. Then $E_1+E_2$ is
    \begin{enumerate}
        \item open if $E_1$ or $E_2$ is open
        \item NOT open unless both $E_1$ and $E_2$ are open
        \item closed if $E_1$ or $E_2$ is closed
        \item closed if both $E_1$ and $E_2$ are closed
    \end{enumerate}
    \item For each $a \in \vec{R}$, consider the linear programming problem Max. $z=x_1+2x_2+3x_3+4x_4$ subject to \\
    $ax_1+2x_3 \leq 1$\\
    $x_1+ax_2+3x_4 \leq 2$\\
    $x_1,x_2,x_3,x_4\geq0$.\\
    Let $S=\cbrak{a \in \vec{R}:\text{ the given LP problem has a basic feasible solution}}.$
    \begin{enumerate}
        \item $S=\phi$
        \item $S=\vec{R}$
        \item $S=\brak{0,\infty}$
        \item $S=\brak{\vec{-\infty,0}}$
    \end{enumerate}
    \item Consider the linear programming problem Max. $z=x_1+5x_2+3x_3$ subject to\\
    $2x_1-3x_2+5x_3\leq3$\\
    $3x_1+2x_3\leq5$\\
    $x_1,x_2,x_3\geq0$.\\
    Then the dual of this LP problem
    \begin{enumerate}
        \item has a feasible solution but does NOT have a basic feasible solution 
        \item has a basic feasible solution
        \item has infinite number of feasible solutions
        \item has no feasible solution
    \end{enumerate}
    \item Consider a transportation problem with two warehouses and two markets. The warehouse capacities are $a_1=2$ and $a_2=4$ and the market demands $b_1=3$ and $b_2=3$. Let $x_n$ be the quantity shipped from warehouse $i$ to market $j$ and $c_{ij}$ be the corresponding unit cost. Suppose that $c_{11}=1,c_{21}=1,c_{22}=2$. $\brak{x_{11},x_{12},x_{21},x_{22}}=\brak{2,0,1,3}$ is optimal for every 
    \begin{enumerate}
        \item $c_{12}=\sbrak{1,2}$
        \item $c_{12}=\sbrak{0,3}$
        \item $c_{12}=\sbrak{1,3}$
        \item $c_{12}=\sbrak{2,4}$
    \end{enumerate}
    \item The smallest degree of the polynomial that interpolates the data\\
    \begin{table}[h!]
        \centering
        \begin{tabular}{|c|c|c|c|c|c|c|}
\hline
    $x$ & $-2$ &$-1$ & $0$ & $1$ & $2$ & $3$  \\
\hline
    $f\brak{x}$ & $-58$ & $-21$ & $-12$ & $-13$ & $-6$ & $27$\\
\hline
\end{tabular}

    \end{table}\\
    is
    \begin{enumerate}
        \item $3$
        \item $4$
        \item $5$
        \item $6$
    \end{enumerate}
    \item Suppose that $x_0$ is sufficiently close to $3$. Which of the following iterations $x_{n+1}=g\brak{x_n}$ will converge to the fixed point $x=3$?
    \begin{enumerate}
        \item $x_{n+1}=-16+6x_n+\frac{3}{x_n}$
        \item $x_{n+1}=\sqrt{3+2x_n}$
        \item $x_{n+1}=\frac{3}{x_n-2}$
        \item $x_{n+1}=\frac{x_n^2-3}{2}$
    \end{enumerate}
    
    
     
\end{enumerate}





\end{document}

